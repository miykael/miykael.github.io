% -- Encoding UTF-8
\documentclass[english]{cv-style}

\usepackage{graphicx}

\begin{document}

\header{Michael P. }{Notter}
\lastupdated

%----------------------------------------------------------------------------------------
% SIDEBAR SECTION  -- In the aside, each new line forces a line break
%----------------------------------------------------------------------------------------
\begin{aside}
\section{~}
\includegraphics[width=4cm]{profile_pic_michael}
%
\section{Contact}
Tel: +41 (0)79 786 47 17
\href{mailto:michaelnotter@hotmail.com}{michaelnotter@hotmail.com}
~
Chemin de la Millière 1
1040 Villars-le-Terroir
Switzerland
%
~
\section{Method Skills}
Neuroimaging (MRI \& EEG)
Signal Processing
Machine Learning
Statistical Analysis
Data Presentation
Open Source Development
%
~
\section{Computer Skills}
Python, Shell, MATLAB
Linux, Windows, macOS
Version control (git)
Docker, Singularity
Testing (CircleCI, Travis)
%
~
\section{Languages }
German (native)
English (C2)
French (C1)
%
~
\section{Find me also on}
Github: \href{https://github.com/miykael}{github.com/miykael}
Google Scholar: \href{https://scholar.google.com/citations?user=cXB2flkAAAAJ&hl=en}{goo.gl/yft8px}
LinkedIn: \href{https://www.linkedin.com/in/michael-notter-5542b464/}{michael-notter}
Twitter: \href{https://twitter.com/miyka_el}{twitter.com/miyka\_el}
Web: \href{https://miykael.github.io}{https://miykael.github.io}
%
\end{aside}

%----------------------------------------------------------------------------------------
% RESUME SECTION
%----------------------------------------------------------------------------------------
\vspace{0.2cm}
\section{About me}
  \vspace{-0.2cm}
I'm a data scientist working at the intersection of neuroscience \& computer science. My love for programming, methods, and the brain are best shown in my involvement in many open source projects (leading and collaborating), covering many different signal domains (e.g. MRI, EEG, behavioral). I enjoy working on challenging projects in a vibrant and stimulating community, am an enthusiastic learner, a motivated teacher and always curious \& excited to learn something new.\\
~
%----------------------------------------------------------------------------------------
% EDUCATION SECTION
%----------------------------------------------------------------------------------------
\section{Education}
\begin{entrylist}
\entry
  {\scalebox{.8}[1.0]{04/2016 - present}}
  {PhD in Neuroscience}
  {\jobtitle{University of Lausanne, Switzerland}}
  {Investigation of multisensory integration in the human brain by means of time-resolved functional magnetic resonance imaging (fMRI). Analyses incorporate standard univariate approaches, as well as machine learning approaches, such as multivariate pattern analysis (MVPA) \& convoluted neural networks.\\}
\end{entrylist}
\begin{entrylist}
\entry
  {\scalebox{.8}[1.0]{02/2012 - 07/2014}}
  {Master of Science in Cognitive Psychology \& Neuroscience}
  {\jobtitle{University of Zurich, Switzerland}}
  {Major in Psychology \& minor in Neuroinformatics, covering neurobiology, cognitive psychology, neuroimaging methods, neural networks, models of computation \& computational vision.\\}
\end{entrylist}
\begin{entrylist}
\entry
  {\scalebox{.8}[1.0]{09/2007 - 02/2012}}
  {Bachelor of Science in Psychology}
  {\jobtitle{University of Zurich, Switzerland}}
  {Major in Psychology \& minor in Neuroinformatics, covering psychology, statistics, neuroscience, informatics, biology, mathematics \& artificial intelligence.}
\end{entrylist}

%----------------------------------------------------------------------------------------
% PROFESSIONAL EXPERIENCE SECTION
%----------------------------------------------------------------------------------------
\section{Professional Experience}
\begin{entrylist}
\entry
  {\scalebox{.8}[1.0]{04/2014 - 04/2016}}
  {Research Collaborator in Neuroscience \& Neuroimaging}
  {\jobtitle{CHUV, Lausanne, Switzerland}}
  {Development, implementation, and analysis of behavioral \& neuroimaging experiments using eye-tracking, EEG \& MRI methods. Technical support \& teaching of collaborators, and development of MRI, EEG \& eye-tracking analysis software.\\}
\end{entrylist}
\begin{entrylist}
\entry
  {\scalebox{.8}[1.0]{02/2013 - 11/2014}}
  {Research Assistant in Neuroscience \& Neuroimaging}
  {\jobtitle{INAPIC, Zurich, Switzerland}}
  {Development and maintenance of scripts for the analysis of behavioral, physiological and neuroimaging (MRI) data.\\Extensive support of research collaborators in their data analysis.\\}
\end{entrylist}
\begin{entrylist}
\entry
  {\scalebox{.8}[1.0]{01/2012 - 12/2017}}
  {Special officer in the Psychological-Pedagogical Service of the Armed Forces}
  {\jobtitle{Switzerland}}
  {Counseling, stress prevention \& guidance to soldiers \& cadre of the Armed Forces.\\}
\end{entrylist}
\begin{entrylist}
\entry
  {\scalebox{.8}[1.0]{01/2011 - 05/2011}}
  {Internship at Massachusetts Institute of Technology}
  {\jobtitle{MIT, Cambridge, USA}}
  {Design and execution of experiments, technical support, teaching \& implementation and maintenance of analysis software.\\Internship was prolongated for a month due to very satisfactory work.\\}
\end{entrylist}
\begin{entrylist}
\entry
  {\scalebox{.8}[1.0]{03/2007 - 03/2014}}
  {Clerk in payment transaction at Migros Bank}
  {\jobtitle{Zurich, Switzerland}}
  {Data analysis for the purpose of anomaly detection and process optimization.\\}
\end{entrylist}

\newgeometry{left=1.0cm, bottom=1cm, top=1cm, right=1cm}
\newpage

%----------------------------------------------------------------------------------------
% Publication SECTION
%----------------------------------------------------------------------------------------

\section{Publications}

\large\textbf{Papers}\normalsize

\begin{publist}
\publications
  {\scalebox{.8}[1.0]{2018}}
  {\textbf{Notter, M.}, Gabrieli, J.D.E., \& Geiser, E. (under review). Neural correlates of individual differences in temporal Gestalt perception. Frontiers in Neuroscience.}
\publications
  {\scalebox{.8}[1.0]{}}
  {\textbf{Notter, M.P.}, Hanke, M., Murray, M.M., \& Geiser, E. (2018). Encoding of Auditory Temporal Gestalt in the Human Brain. \color{gray}\bodyfontit{Cerebral Cortex, 1}, 10. \href{https://doi.org/10.1093/cercor/bhx328}{https://doi.org/10.1093/cercor/bhx328}}
\publications
  {\scalebox{.8}[1.0]{2017}}
  {Crottaz-Herbette, S., Fornari, E., \textbf{Notter, M.P.}, Bindschaedler, C., Manzoni, L., \& Clarke, S. (2017). Reshaping the brain after stroke: the effect of prismatic adaptation in patients with right brain damage. \color{gray}\bodyfontit{Neuropsychologia}, 104, 54-63. \href{https://doi.org/10.1016/j.neuropsychologia.2017.08.005}{https://doi.org/10.1016/j.neuropsychologia.2017.08.005}}
\publications
  {\scalebox{.8}[1.0]{}}
  {Zeugin, D., Arfa, N., \textbf{Notter, M.}, Murray, M.M., \& Ionta, S. (2017). Implicit self-other discrimination affects the interplay between multisensory affordances of mental representations of faces. \color{gray}\bodyfontit{Behavioural brain research}, 333, 282-285. \href{https://doi.org/10.1016/j.bbr.2017.06.044}{https://doi.org/10.1016/j.bbr.2017.06.044}}
\publications
  {\scalebox{.8}[1.0]{2012}}
  {Geiser, E., \textbf{Notter, M.}, \& Gabrieli, J.D.E. (2012). A corticostriatal neural system enhances auditory perception through temporal context processing. \color{gray}\bodyfontit{The Journal of Neuroscience, 32(18)}, 6177‐6182. \href{https://doi.org/10.1523/JNEUROSCI.5153-11.2012}{https://doi.org/10.1523/JNEUROSCI.5153-11.2012}}
\end{publist}

\large\textbf{Abstracts}\normalsize

\begin{publist}
\publications
  {\scalebox{.8}[1.0]{2018}}
  {\textbf{Notter, M.P.}, Herholz, P., \& Murray, M.M. (2018). fmriflows - a consortium of fully autonomous univariate and multivariate fMRI pro-cessing pipelines. \color{gray}\bodyfontit{Poster at Lemanic Neuroscience Annual Meeting (LNAM), September 2-3, 2018. \href{https://doi.org/10.6084/m9.figshare.7028168}{https://doi.org/10.6084/m9.figshare.7028168}}}
\publications
  {\scalebox{.8}[1.0]{}}
  {Herholz, P., Gulban, O.F., Ernst, M. \& \textbf{Notter, M.} (2018). ALPACA - Automatic Localization and Parcellation of Auditory Cortex Areas. \color{gray}\bodyfontit{Poster at Neurohackademy 2018, July 31, 2018. \href{https://doi.org/10.6084/m9.figshare.6938336}{https://doi.org/10.6084/m9.figshare.6938336}}}
\publications
  {\scalebox{.8}[1.0]{2017}}
  {\textbf{Notter, M.P.}, Hanke, M., Murray, M.M., \& Geiser, E. (2017). Temporal Gestalt perception: Where in the human brain do we encode rhythm?. \color{gray}\bodyfontit{Poster at Lemanic Neuroscience Annual Meeting (LNAM), September 1-2, 2017. \href{https://doi.org/10.6084/m9.figshare.6989696}{https://doi.org/10.6084/m9.figshare.6989696}}}
\publications
  {\scalebox{.8}[1.0]{2016}}
  {\textbf{Notter, M.P.}, Hanke, M., Murray, M.M., \& Geiser, E. (2016). Where the rhythm plays: Machine learning decodes rhythm-sensitive cortices. \color{gray}\bodyfontit{Poster at 22nd Annual Meeting of the Organization for Human Brain Mapping, June 27-30. \href{https://doi.org/10.6084/m9.figshare.6989681.v1}{https://doi.org/10.6084/m9.figshare.6989681}}}
\publications
  {\scalebox{.8}[1.0]{2012}}
  {Gorgolewski, K., Halchenko, Y., Hanke, M., \textbf{Notter, M.}, Varoquaux, G., Waskom, M.L., Ziegler, E., \& Ghosh, S. (2012) Studying resting state connectivity using Nipype. \color{gray}\bodyfontit{Poster at third biennial resting state conference, Magdeburg, Germany, Sep 5-7. \href{https://doi.org/10.6084/m9.figshare.6989672.v1}{https://doi.org/10.6084/m9.figshare.6989672}}}
\publications
  {\scalebox{.8}[1.0]{}}
  {Gorgolewski, K., Halchenko, Y., Hanke, M., \textbf{Notter, M.}, Varoquaux, G., Waskom, M., Ziegler, E., \& Ghosh, S. (2012) Nipype 2012: more packages, reusable workflows and reproducible science. \color{gray}\bodyfontit{Poster at 18th Annual Meeting of the Organization for Human Brain Mapping, June 10-14. \href{https://doi.org/10.6084/m9.figshare.6989663}{https://doi.org/10.6084/m9.figshare.6989663}}}
\end{publist}

\large\textbf{Open Source Software}\normalsize

\begin{publist}
\publications
  {\scalebox{.8}[1.0]{2018}}
  {Knebel, J.-F. \& \textbf{Notter, M.P.} (2018). STEN 2.0: Statistical Toolbox for Electrical Neuroimaging. \color{gray}\bodyfontit{Zenodo. \href{https://doi.org/10.5281/zenodo.1164151}{https://doi.org/10.5281/zenodo.1164151}}}
\publications
  {\scalebox{.8}[1.0]{2017}}
  {Kirstie, W., \textbf{Notter, M.P.} \& Morgan, S. (2017). Brains for Publication. \color{gray}\bodyfontit{Zenodo. \href{https://doi.org/10.5281/zenodo.596845}{https://doi.org/10.5281/zenodo.596845}}}
\publications
  {\scalebox{.8}[1.0]{}}
  {\textbf{Notter, M.P.}, \& Murray, M.M. (2017). Temporal Binding Window scripts: a lightweight matlab tool to analyse the temporal binding window in a multisensory integration study. \color{gray}\bodyfontit{Zenodo. \href{https://doi.org/10.5281/zenodo.815876}{https://doi.org/10.5281/zenodo.815876}}}
\publications
  {\scalebox{.8}[1.0]{}}
  {\textbf{Notter, M.P.}, \& Murray, M.M. (2017). Pupillometry Analyzer: a lightweight matlab tool to pre-process pupillometry data. \color{gray}\bodyfontit{Zenodo. \href{https://doi.org/10.5281/zenodo.592386}{https://doi.org/10.5281/zenodo.592386}}}
\publications
  {\scalebox{.8}[1.0]{}}
  {\textbf{Notter, M.P.}, Knebel, J.-F. \& Murray, M.M. (2017). LINEViewer: a Python based EEG analysis tool that provides a rough data overview. \color{gray}\bodyfontit{Zenodo. \href{https://doi.org/10.5281/zenodo.593318}{https://doi.org/10.5281/zenodo.593318}}}
\publications
  {\scalebox{.8}[1.0]{2016}}
  {Gorgolewski, K.J., Esteban, O., Ziegler, E., \textbf{Notter, M.P.}, … Ghosh, S. (2016). Nipype: a flexible, lightweight and extensible neuroimaging data processing framework in Python. \color{gray}\bodyfontit{Zenodo. \href{https://doi.org/10.5281/zenodo.596855}{https://doi.org/10.5281/zenodo.596855}}}
\publications
  {\scalebox{.8}[1.0]{2015}}
  {Halchenko, Y., Hanke, M., Oosterhof, N.N., Olivetti, E., Sederberg, P.B., Guntupalli, S., … \textbf{Notter, M.} … \& Ma, F. (2015). PyMVPA: 2.4.1. \color{gray}\bodyfontit{Zenodo. \href{http://doi.org/10.5281/zenodo.33988}{http://doi.org/10.5281/zenodo.33988}}}
\end{publist}

\newpage

%----------------------------------------------------------------------------------------
% Professional Activities SECTION
%----------------------------------------------------------------------------------------

\section{Professional Activities \& Teaching}
\begin{talklist}
\talks
  {\scalebox{.8}[1.0]{09/2018}}
  {Neuroimaging in Python}
  {University of Cambridge, UK}
  {2 day workshop - \jobtitle{\href{https://github.com/miykael/workshop\_cambridge}{https://github.com/miykael/workshop\_cambridge}}\\
Workshop given at Cognition \& Brain Sciences Unit covering many different neuroimaging topics, such as task-fMRI, diffusion imaging, functional connectivity analysis, machine learning, convoluted neural networks \& Nipype.}
\end{talklist}

\begin{talklist}
\talks
  {\scalebox{.8}[1.0]{08/2018}}
  {Brainhack Computing: Hands on in Python}
  {Sardar Patel Institute of Technology, Mumbai, India}
  {5 hour webinar - \jobtitle{\href{https://github.com/miykael/workshop\_mumbai}{https://github.com/miykael/workshop\_mumbai}}\\
Webinar given during Brainhack event organized by Prof. Preeti Jani, sponsored by IEEE, covering basics of neuroimaging data analysis using python toolboxes such as Nipype, Nilearn \& Keras.}
\end{talklist}

\begin{talklist}
\talks
  {\scalebox{.8}[1.0]{05/2018}}
  {Open and Reproducible Neuroscience using Python}
  {Max Planck Institute, Frankfurt, Germany}
  {3 day workshop - \jobtitle{\href{https://openreproneuro2018frankfurt.github.io}{https://openreproneuro2018frankfurt.github.io}}\\
Focused on open and reproducible neuroscience using python. Teaching covered scientific toolboxes such as Nipype, Docker, Jupyter Notebook, BIDS, OpenNeuro, DataLad, Nibabel, Nilearn, and PyMVPA.}
\end{talklist}

\begin{talklist}
\talks
  {\scalebox{.8}[1.0]{05/2018}}
  {Neuroimaging with Nipype - Where are we and where are we going?}
  {Otto von Guericke University Magdeburg, Germany}
  {1 hour talk - \jobtitle{\href{https://brainhack.psychoinformatics.de}{https://brainhack.psychoinformatics.de}}\\
Nipype Tutorial given during the Brainhack Global 2018.}
\end{talklist}

\begin{talklist}
\talks
  {\scalebox{.8}[1.0]{03/2018}}
  {Open and Reproducible Neuroscience using Python}
  {University of Marburg, Germany}
{3 day workshop - \jobtitle{\href{https://openreproneuro2018marburg.github.io}{https://openreproneuro2018marburg.github.io}}\\
Workshop covered the same content as at Max Planck Institute, Frankfurt, Germany in May 2018.}
\end{talklist}

\begin{talklist}
\talks
  {\scalebox{.8}[1.0]{03/2017}}
  {Nipype Tutorial: How to analyze your MRI data in an easy and flexible way}
  {University of Zurich, Switzerland}
  {2 hour talk - \jobtitle{\href{https://dynage.github.io/brainhack-zh}{https://dynage.github.io/brainhack-zh}}\\
Nipype Tutorial given during the Brainhack Global 2017.}
\end{talklist}

\begin{talklist}
\talks
  {\scalebox{.8}[1.0]{03/2017}}
  {Nipype Tutorial}
  {Global (used in +85 countries)}
  {autodidactic teaching tool - \jobtitle{\href{https://miykael.github.io/nipype\_tutorial}{https://miykael.github.io/nipype\_tutorial}}\\
New and improved user's guide that uses Docker and Jupyter Notebooks for an interactive introduction to Nipype and related neuroimaging software. The homepage is visited more than 2'500 times per month.}
\end{talklist}

\begin{talklist}
\talks
  {\scalebox{.8}[1.0]{08/2011}}
  {Nipype Beginner's Guide}
  {Global (used in +130 countries)}
  {autodidactic teaching tool - \jobtitle{\href{http://miykael.github.io/nipype-beginner-s-guide}{http://miykael.github.io/nipype-beginner-s-guide}}\\
First comprehensive guide to Nipype with more than 3'500 visits per month.}
\end{talklist}

%----------------------------------------------------------------------------------------
% Awards & Fellowships SECTION
%----------------------------------------------------------------------------------------

\section{Awards \& Fellowships}
\begin{publist}

\publications
  {\scalebox{.8}[1.0]{2018}}
  {Invitation to 3-day code sprint at Massachusetts Institute of Technology (MIT).}

\publications
  {\scalebox{.8}[1.0]{2018}}
  {SSN Travel Fellowships for Student \& Postdoc Members for 1'500.00 CHF.}

\publications
  {\scalebox{.8}[1.0]{2018}}
  {Chosen from 400 applicants to be one of 60 participants at the Neurohackademy 2018 in Seattle, a two-week hands-on summer school in neuroimaging and data science.}

\end{publist}

%----------------------------------------------------------------------------------------
% References SECTION
%----------------------------------------------------------------------------------------

\section{References}
\begin{reflist}
\reference
  {\scalebox{1.0}[1.0]{Satrajit Ghosh}}
  {Principal investigator at MIT \& assistant professor at Harvard Medical School (\href{mailto:satra@mit.edu}{satra@mit.edu})}

\reference
  {\scalebox{1.0}[1.0]{Michael Hanke}}
  {Junior\-professor \& principal investigator at university of Magdeburg (\href{mailto:michael.hanke@ovgu.de}{michael.hanke@ovgu.de})}

\reference
  {\scalebox{1.0}[1.0]{Eveline Geiser}}
  {Principal investigator at CHUV (\href{mailto:eveline.geiser@unil.ch}{eveline.geiser@unil.ch})}

\end{reflist}

%----------------------------------------------------------------------------------------
\end{document}
